\documentclass[11pt]{article}

\usepackage{a4wide,times}
\usepackage[english]{babel}

% -----------------------------------------------
% especially use this for you code
% -----------------------------------------------

\usepackage{courier}
\usepackage{listings}
\usepackage{color}
\usepackage{tabularx}

\definecolor{Gray}{gray}{0.95}

\definecolor{mygreen}{rgb}{0,0.6,0}
\definecolor{mygray}{rgb}{0.5,0.5,0.5}
\definecolor{mymauve}{rgb}{0.58,0,0.82}

\lstset{language=C++,
	basicstyle = \normalsize\ttfamily,   % the size and fonts that are used
	tabsize = 2,                    % sets default tabsize
	breaklines = true,              % sets automatic line breaking
	keywordstyle=\color{blue}\ttfamily,
	stringstyle=\color{red}\ttfamily,
	commentstyle=\color{mygreen}\ttfamily,
	keepspaces=true,
	showspaces=false,
	showstringspaces=false,
}

\lstdefinestyle{bigtabs}
{
	basicstyle = \normalsize\ttfamily,
	tasize = 8;
}

\begin{document}

\title{Programming report \\
       Week 5 Assignments C++
}
\date{\today}
\author{Jaime Betancor Valado \\
Christiaan Steenkist \\
Remco Bos \\
Diego Ribas Gomes
}

\maketitle

\section*{Assignment 40, \texttt{strcpy}}
We were tasked with comparing a pointer and index implementations of \texttt{strcpy}

\subsection*{How pointer implementation works}
To the function a C-string is passed as a pointer\texttt{src} pointing to a const char.
The \texttt{src} pointer assigns the character at the current incrementation to the same
incrementation of \texttt{dst} (i.e. first element where \texttt{src} is pointing at is assigned to the first element
where \texttt{dst} is pointing at, the next incrementation copies the second element, etc).
The loop terminates after the end of the C-string is reached, i.e. when \texttt{'\textbackslash0'}.

\subsection*{How index-using implementation works}
An empty char array \texttt{dst} must be provided where the \texttt{src} char array can copy its elements to.
In reality the arrays are pointers to a memory location where the characters are stored(\texttt{src}) or gets copied to(\texttt{dst}).
The incrementation variable is instantiated as 0 to start at the first element. In the while-loop the first element
of \texttt{dst} is assigned the value of the first element of \texttt{src}. Then the loop increments the index variable to copy the other
elements to \texttt{dst}. Just like the pointer version, the while-loop terminates when the \texttt{'\textbackslash0'} character is reached.

\subsection{Why pointers are preferred}
It takes less code to do the pointer version, so it is also easier to read the code.
It is faster and more efficient in memory, because only a memory address has to be provided instead of a whole array.
This also gives pointers the ability to allocate memory dynamically, because it uses the elements at the address one by one.
If an array is used, than all its elements have to be provided, then one by one the elements are copied(\texttt{src}) to the other array(\texttt{dst})
which is empty at default before copying. So the array is copied and memory must be allocated for it. 
The pointer version points at the element(\texttt{src}) in memory and assigns it to the memory position of the other pointer(\texttt{dst}).
So using an array and iterating over its elements is slower and less efficient than using a pointer that take the element directly from memory.


\subsection*{Code listings}
\lstinputlisting[caption = \tt example.cc]{src/a40}

\section*{Assignment 41, new and delete}
Here are the variants of new/delete.

\subsection*{1 new/delete}
To allocate primitive types or objects; eg:
\begin{lstlisting}
	int *v1 = new int;		// an int pointer variable points to memory allocated by operator new
	string *s1 = new string;	// a class-type object is allocated
\end{lstlisting}

To release the memory of a single element allocated using new; eg:
\begin{lstlisting}
	delete v1;
	delete s1;
\end{lstlisting}

\subsection*{2 new[]/delete[]}
To allocate dynamic arrays, whose lifetime may exceed the lifetime of the function in which they were created; eg:
\begin{lstlisting}
	int *intarr = new int[20];		// allocates 20 ints
	string *stringarr = new string[10];	// allocates 10 class-type objects 'string'
\end{lstlisting}

To call the destructor for each element in the array and return the memory pointed at by the pointer to the common pool; eg:
\begin{lstlisting}
	delete[] intarr;
	delete[] stringarr;
\end{lstlisting}

\subsection*{3 operator new(sizeInBytes)/operator delete}
To allocate raw memory, a block of memory for unspecified purpose; eg:
\begin{lstlisting}
	// the raw memory returned by new is a void *, here assigned to a string * variable
	string *stPtr = static_cast<string *>(operator new(10 * sizeof(std::string)));
\end{lstlisting}

To return the raw memory allocated by operator new; eg:
\begin{lstlisting}
	operator delete(chPtr);
\end{lstlisting}

\subsection*{4 placement new}
To initialize an object or value into an existing block of memory; eg:
\begin{lstlisting}
	//installs a string into the block of memory created in the example from section 3.1
	string *sp = new(chPtr) std::string("example");
\end{lstlisting}

The destructor has to be called explicitly to destroy the object, followed by operator delete to deallocate the memory as in section 3.2, eg:
\begin{lstlisting}	
	sp->~string();
\end{lstlisting}

\section*{Assignment 42, \texttt{Strings} destructor}
We were tasked with making a destructor for the \texttt{Strings} class.

\subsection*{Code listings}

\lstinputlisting[caption = \tt strings.h]{src/a42/strings/strings.h}
\lstinputlisting[caption = \tt destructor.cc]{src/a42/strings/destructor.cc}

\section*{Assignment 43, double pointers}
We were tasked with changing the \texttt{Strings} from single pointer to double pointer.

\subsection*{Code listings}
\lstinputlisting[caption = \tt strings.h]{src/a43/strings/strings.h}
\lstinputlisting[caption = \tt at1.cc]{src/a43/strings/at1.cc}
\lstinputlisting[caption = \tt at2.cc]{src/a43/strings/at2.cc}
\lstinputlisting[caption = \tt addstring.cc]{src/a43/strings/addstring.cc}
\lstinputlisting[caption = \tt capacity.cc]{src/a43/strings/capacity.cc}
\lstinputlisting[caption = \tt destructor.cc]{src/a43/strings/destructor.cc}
\lstinputlisting[caption = \tt reserve.cc]{src/a43/strings/reserve.cc}
\lstinputlisting[caption = \tt resize.cc]{src/a43/strings/resize.cc}
\lstinputlisting[caption = \tt rawpointers.cc]{src/a43/strings/rawpointers.cc}

\section*{Assignment 44, placement new}
We were tasked with setting up the \texttt{Strings} class using placement new.

\subsection*{Code listings}
\lstinputlisting[caption = \tt strings.h]{src/a44/strings/strings.h}
\lstinputlisting[caption = \tt at1.cc]{src/a44/strings/at1.cc}
\lstinputlisting[caption = \tt at2.cc]{src/a44/strings/at2.cc}
\lstinputlisting[caption = \tt addstring.cc]{src/a44/strings/addstring.cc}
\lstinputlisting[caption = \tt capacity.cc]{src/a44/strings/capacity.cc}
\lstinputlisting[caption = \tt destructor.cc]{src/a44/strings/destructor.cc}
\lstinputlisting[caption = \tt reserve.cc]{src/a44/strings/reserve.cc}
\lstinputlisting[caption = \tt resize.cc]{src/a44/strings/resize.cc}
\lstinputlisting[caption = \tt rawpointers.cc]{src/a44/strings/rawstrings.cc}

\end{document}