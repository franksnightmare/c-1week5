\documentclass[11pt]{article}

\usepackage{a4wide,times}
\usepackage[english]{babel}

% -----------------------------------------------
% especially use this for you code
% -----------------------------------------------

\usepackage{courier}
\usepackage{listings}
\usepackage{color}
\usepackage{tabularx}

\definecolor{Gray}{gray}{0.95}

\definecolor{mygreen}{rgb}{0,0.6,0}
\definecolor{mygray}{rgb}{0.5,0.5,0.5}
\definecolor{mymauve}{rgb}{0.58,0,0.82}

\lstset{language=C++,
	basicstyle = \normalsize\ttfamily,   % the size and fonts that are used
	tabsize = 2,                    % sets default tabsize
	breaklines = true,              % sets automatic line breaking
	keywordstyle=\color{blue}\ttfamily,
	stringstyle=\color{red}\ttfamily,
	commentstyle=\color{mygreen}\ttfamily,
	keepspaces=true,
	showspaces=false,
	showstringspaces=false,
}

\lstdefinestyle{bigtabs}
{
	basicstyle = \normalsize\ttfamily,
	tasize = 8;
}

\begin{document}

\title{Programming report \\
       Week 5 Assignments C++
}
\date{\today}
\author{Jaime Betancor Valado \\
Christiaan Steenkist \\
Remco Bos \\
Diego Ribas Gomes
}

\maketitle

\section*{Assignment 40, \texttt{strcpy}}
We were tasked with comparing a pointer and index implementations of \texttt{strcpy}

\subsection*{How \texttt{strcpy} works}
To the function a C-string is passed as a pointer \texttt{src} pointing to a const char.
A new pointer variable (pointer \texttt{dst} pointing to a char) which is empty as default
takes the character where \texttt{src} is pointing to as first. 
The while loop iterates over all characters where \texttt{src} is pointing at.
Both pointer variables are incremented and the characters of \texttt{src} are copied
one by one to the \texttt{dst} pointer variable. The incrementation of the pointer
accesses the next character in the memory block. So first the first character
in memory is copied to the first memory place of \texttt{dst}, then the second, etc.
The loop terminates after the end of the C-string is reached, i.e. when \texttt{'\textbackslash0'}
is copied to \texttt{dst}.

\subsection*{How nowhere works}
A C-string is passed to the pointer variable \texttt{src} like the \texttt{strcpy} function.
And a character(or multiple characters) is chosen by the user to see if it is
nowhere in the C-string.
This is done by checking if the character in memory given by the user is equal
to a character in the C-string. Like \texttt{strcpy} the while loop terminates after \texttt{'\textbackslash0'}
is reached, and in this case it also terminates if the character is found, hence it
is not nowhere in the C-string. If the character given by the user is nowhere in the
C-string, an if-statement is used to check if the \texttt{'\textbackslash0'} is reached, hence end of the
C-string and the character is nowhere in the C-string.

\subsection{Why pointer-implementation is preferred}
The pointer-implementation is preferred, because it is more efficient.
If indices are used, then the program first goes to the index and then
looks at the memory. A pointer goes directly to the memory and then performs
its duty.

\subsection*{Code listings}
\lstinputlisting[caption = \tt example.cc]{src/a40}

\section*{Assignment 41, new and delete}
Here are the various new and delete operators that can be used.

\subsection{\texttt{new}}
To allocate primitive types or objects;
\begin{lstlisting}
	int *v1 = new int;		// an int pointer variable points to memory allocated by operator new
	string *s1 = new string;	// a class-type object is allocated
\end{lstlisting}

\subsection{\texttt{delete}}
To release the memory of a single element allocated using new;
\begin{lstlisting}
	delete v1;
	delete s1;
\end{lstlisting}

\subsection{\texttt{new[]}}
To allocate dynamic arrays, whose lifetime may exceed the lifetime of the function in which they were created;

\begin{lstlisting}
	int *intarr = new int[20];		// allocates 20 ints
	string *stringarr = new string[10];	// allocates 10 class-type objects 'string'
\end{lstlisting}

\subsection{\texttt{delete[]}}
To call the destructor for each element in the array and return the memory pointed at by the pointer to the common pool;
\begin{lstlisting}
	delete[] intarr;
	delete[] stringarr;
\end{lstlisting}

\subsection{\texttt{operator new(sizeInBytes)}}
To allocate raw memory, a block of memory for unspecified purpose;
\begin{lstlisting}	
	char *chPtr = static_cast<char *>(operator new(numberOfBytes)); // the raw memory returned by new is a void *, here assigned to a char * variable
\end{lstlisting}

\subsection{\texttt{operator delete}}
To return the raw memory allocated by operator new;
\begin{lstlisting}
	operator delete(chPtr);
\end{lstlisting}

\subsection{\texttt{placement new}}
To initialize an object or value into an existing block of memory;
(Memory allocated this way is returned by explicitly calling the object's destructor)
\begin{lstlisting}
	Type *new(void *memory) Type(arguments);	//memory is a block of memory of at least sizeof(Type) bytes and Type(arguments) is any constructor of the class Type;
\end{lstlisting}

\section*{Assignment 42, \texttt{Strings} destructor}
We were tasked with making a destructor for the \texttt{Strings} class.

\subsection*{Code listings}

\lstinputlisting[caption = \tt strings.h]{src/a42/strings/strings.h}
\lstinputlisting[caption = \tt destructor.cc]{src/a42/strings/destructor.cc}

\section*{Assignment 43, double pointers}
We were tasked with changing the \texttt{Strings} from single pointer to double pointer.

\subsection*{Code listings}
\lstinputlisting[caption = \tt strings.h]{src/a43/strings/strings.h}
\lstinputlisting[caption = \tt at1.cc]{src/a43/strings/at1.cc}
\lstinputlisting[caption = \tt at2.cc]{src/a43/strings/at2.cc}
\lstinputlisting[caption = \tt addstring1.cc]{src/a43/strings/addstring1.cc}
\lstinputlisting[caption = \tt addstring2.cc]{src/a43/strings/addstring2.cc}
\lstinputlisting[caption = \tt capacity.cc]{src/a43/strings/capacity.cc}
\lstinputlisting[caption = \tt destructor.cc]{src/a43/strings/destructor.cc}
\lstinputlisting[caption = \tt reserve.cc]{src/a43/strings/reserve.cc}
\lstinputlisting[caption = \tt resize.cc]{src/a43/strings/resize.cc}

\section*{Assignment 44, placement new}
We were tasked with setting up the \texttt{Strings} class using placement new.

\subsection*{Code listings}
\lstinputlisting[caption = \tt strings.h]{src/a44/strings/strings.h}
\lstinputlisting[caption = \tt at1.cc]{src/a44/strings/at1.cc}
\lstinputlisting[caption = \tt at2.cc]{src/a44/strings/at2.cc}
\lstinputlisting[caption = \tt addstring1.cc]{src/a44/strings/addstring1.cc}
\lstinputlisting[caption = \tt addstring2.cc]{src/a44/strings/addstring2.cc}
\lstinputlisting[caption = \tt capacity.cc]{src/a44/strings/capacity.cc}
\lstinputlisting[caption = \tt destructor.cc]{src/a44/strings/destructor.cc}
\lstinputlisting[caption = \tt reserve.cc]{src/a44/strings/reserve.cc}
\lstinputlisting[caption = \tt resize.cc]{src/a44/strings/resize.cc}

\end{document}